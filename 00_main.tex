\documentclass[conference]{IEEEtran}

\IEEEoverridecommandlockouts
% The preceding line is only needed to identify funding in the first footnote. If that is unneeded, please comment it out.
\usepackage{cite}
\usepackage{amsmath,amssymb,amsfonts}
\usepackage{algorithmic}
\usepackage{graphicx}
\usepackage{textcomp}
\usepackage{xcolor}
\def\BibTeX{{\rm B\kern-.05em{\sc i\kern-.025em b}\kern-.08em
    T\kern-.1667em\lower.7ex\hbox{E}\kern-.125emX}}
\begin{document}

\title{Coordinated Voltage Control in Distribution
Systems with Distributed Generations}

\author{\IEEEauthorblockN{Alvi Newaz}
\IEEEauthorblockA{Electrical and Computer Engineering\\
Florida State University\\
Tallahassee, Florida\\
Email: an15m@fsu.edu}
\and
\IEEEauthorblockN{Juan Ospina} 
\IEEEauthorblockA{Electrical and Computer Engineering\\
Florida State University\\
Tallahassee, Florida\\
Email: jjospina@fsu.edu}
\and
\IEEEauthorblockN{M. Omar Faruque} 
\IEEEauthorblockA{Electrical and Computer Engineering\\
Florida State University\\
Tallahassee, Florida\\
Email: faruque@caps.fsu.edu}

}

\maketitle

\graphicspath{{figs/}}
\begin{abstract}
This paper presents a centralized coordinated voltage control algorithm for distribution systems with distributed generations (DGs).  The control algorithm coordinates the use of available distributed generation (DG) with the traditional voltage regulation devices, taking into consideration the reactive power limits of the DG inverters to maintain the system voltage within predefined limits. With the help of the developed algorithm the  reactive power generation capacity of the DG inverters can be used to cope with the dynamic generation of the DGs without increasing the operations of the traditional voltage regulation devices available.
\end{abstract}

\begin{IEEEkeywords}
Distributed generations, voltage regulation devices, Coordinated Voltage Control
\end{IEEEkeywords}

\section{Introduction}\label{sec:intro}
The energy industry is rapidly approaching a significant penetration of distributed generation (DG) in the distribution system. This higher penetration of distributed energy resources (DERs) have introduced reverse power flow in the system. The traditional voltage regulation devices only consider unidirectional power flow and operate based on the consideration that the voltage down the line will decrease depending on the line impedance. But this assumption is not valid with distributed generation as they can change the voltage of a node by supplying or absorbing reactive power. The uncertainty of the distributed wind and solar renewable resources can also cause traditional voltage regulation devices to operate more than necessary \cite{int1}. 

With the rise of technology such as smart grid, advanced metering infrastructure (AMI) and phasor measurement units (PMUs), a lot more data regarding the system status is available. How to efficiently use this data to optimize the voltage control coordination is still a contemporary research topic. In most of the current literature this problem is addressed as an optimum power flow problem or a mixed integer optimization problem \cite{int2}. But both these approaches usually propose methods which are very computationally intensive and time consuming. As example in \cite{LR5,LR1,LR2,LR3} researchers describe various methods to coordinate voltage control using the DG and traditional voltage regulation devices. But the control cycles in these methods are in the range of minutes to an hour \cite{LR4}. This makes them unfavourable for controlling the voltage in a distribution grids containing fast changing DGs.

This paper proposes a coordinated voltage control method that reduces the use of traditional voltage control devices and utilizes the reactive power generation capabilities of the DG inverters  to regulate the distribution system voltage.  The proposed algorithm enables cooperation between multiple DGs to control the system voltage. This ensures that the full potential of the DG inverters are utilized before using the traditional devices for voltage regulation. The proposed method also completes it's control cycle within a second which makes it suitable for distribution grids containing fast changing DGs. The proposed method has been validated using a distribution feeder located at Florida with real field data. A phasor model of the system with the control algorithm was simulated in in MATLAB\textsuperscript{\textregistered} Simulink\textsuperscript{\textregistered} using the Simscape Power Systems\textsuperscript{TM} toolbox for validation.

The rest of the paper is organized as follows. Section II contains the theoretical background and proposed algorithm. Section III contains the details about the test feeder. Section IV contains the validation and results while Section VI includes the concluding remarks.

\section{Design of Control Algorithm}\label{sec:design}
The proposed algorithm makes use of two concepts to coordinate the voltage regulation between the nodes. They are:
\begin{itemize}
\item Voltage sensitivity analysis
\item Electrical distance calculation
\end{itemize}

\subsection{Voltage Sensitivity Analysis}
The aim of the voltage sensitivity analysis is to determine the effect on the voltage due to reactive power injection at different nodes. Equation (\ref{eq.Pf_equation}) represents the system power-flow equation.  Here ${\Delta Q}, {\Delta P}, {\Delta |V|}$ and ${\Delta \delta}$ represent the change in real power, reactive power, voltage magnitude and voltage angle of nodes. $J$ represents the system jacobian matrix. The elements of $J$ are shown in (\ref{eq.J_equation})
\begin{equation}\label{eq.Pf_equation}
\begin{bmatrix}
{\Delta P}\\ {\Delta Q}
\end{bmatrix} =J\begin{bmatrix}
{\Delta|V|} \\ {\Delta\delta}
\end{bmatrix}
\end{equation}  
Where,
\begin{equation}\label{eq.J_equation}
    J = \begin{bmatrix}
{J_{P|V|}} & {J_{P\delta}}\\ {J_{Q|V|}} & {J_{Q\delta}}
\end{bmatrix}
\end{equation}

Using the components of the system Jacobian matrix the $J_{VQ}$ and $J_{VP}$ matrices can be formulated using (\ref{eq.JVQ}) and (\ref{eq.JVP}).
\begin{equation}\label{eq.JVQ}
    {J_{VQ}} = {J_{Q|V|}}-{J_{Q\delta}}{J_{P\delta}}^{-1}{J_{P|V|}}
\end{equation}

\begin{equation}\label{eq.JVP}
    {J_{VP}}={J_{P|V|}}-{J_{P\delta}}{J_{Q\delta}}^{-1}{J_{Q|V|}}
\end{equation}

Using (\ref{eq.JVQ}) and (\ref{eq.JVP}) the total voltage sensitivity due to real and reactive power injected by a DG can be calculated using (\ref{eq.V_PQ}) \cite{Th_ali}. Here, $[\Delta|V|_{PQ}] , [P_{PV}] $ and $[Q_{PV}]$ represent the total voltage change, real power injected by DG and reactive power injected by DG respectively.

\begin{equation}\label{eq.V_PQ}
[\Delta|V|_{PQ}] ={ J_{VP}}^{-1} [P_{PV}]+{ J_{VQ}}^{-1} [Q_{PV}]
\end{equation}

Using the relation $[P_{PV}] = \frac{[Q_{PV}]}{\cos^{-1}(p.f)}$ equation (\ref{eq.V_PQ}) can be written as
\begin{equation}\label{eq.V_Q}
 [\Delta|V|_{PQ}] ={ J_{VP}}^{-1}\frac{[Q_{PV}]}{\cos^{-1}(p.f)} +{ J_{VQ}}^{-1} [Q_{PV}]
\end{equation}
 Equation (\ref{eq.V_Q}) gives us the relation between total voltage sensitivity and reactive power injection. Here, '$p.f$' represents power factor.

\subsection{Electrical Distance Calculation}
Electrical distance is calculated in order to determine the relative impact on one node due to change in another node. The electrical distances are calculated by determining $M_{sVQ}$ according to equation (\ref{MsQV}) \cite{int1}.

\begin{equation}\label{MsQV}
\begin{bmatrix}
M_{s\alpha}P & M_{s\alpha}Q \\ 
M_{sVP} & M_{sVQ} 
\end{bmatrix} = J^{-1}
\end{equation}
 Here, $J$ is the system jacobian matrix and $M_{s\alpha}P, M_{s\alpha}Q, M_{sVP}$ and $M_{sVQ}$  are the four elements of the inverse of the system jacobian matrix. Equation (\ref{MsVq_expan}) shows the elements of $M_{sVQ}$. \cite{int1}

\begin{equation}\label{MsVq_expan}
M_{sVQ} =
\begin{bmatrix}
\frac{\delta V_{s1}}{\delta Q_{s1}}  & \frac{\delta V_{s1}}{\delta Q_{s2}} & \cdots & \frac{\delta V_{s1}}{\delta Q_{sn}}\\

\frac{\delta V_{s2}}{\delta Q_{s1}}  & \frac{\delta V_{s2}}{\delta Q_{s2}} & \cdots & \frac{\delta V_{s2}}{\delta Q_{sn}}\\

\vdots & \vdots & \vdots & \vdots\\

\frac{\delta V_{sn}}{\delta Q_{s1}}  & \frac{\delta V_{sn}}{\delta Q_{s2}} & \cdots & \frac{\delta V_{sn}}{\delta Q_{sn}}\\ 
\end{bmatrix}
\end{equation}
 
Here $V_{sn}$ and $Q_{sn}$ represent the voltage and reactive power of the node $n$. The electrical distance $D_{sij}$ between the nodes $i$ and $j$ can be calculated using (\ref{ED}) \cite{int1}. 

\begin{equation}\label{ED}
D_{sij} = -\log (\mu_{sij} * \mu_{sji})
\end{equation}

Where,
$$
\mu_{sij} =  \frac{\delta V_{si}}{\delta Q_{sj}} / \frac{\delta V_{sj}}{\delta Q_{sj}}
$$
After determining the electrical distances the nodes are separated into different zones to determine which DG should provide reactive power support for voltage control.

\subsection{Singular value decomposition and pseudo inverse}
In case of a distribution system the system jacobian matrix $J$ is usually a sparse matrix due to the radial nature of distribution systems. Calculating the inverse of $J$ becomes problematic due to this property. For this reason (\ref{eq.V_Q}) and (\ref{ED}) are implemented by determining the pseudo inverse of $J$ using singular value decomposition. The system jacobian matrix $J$ can factored into the expression shown in (\ref{eq.SVD}) \cite{PINV}.
\begin{equation}\label{eq.SVD}
    J = USV^T
\end{equation}

Here, $U$ is an orthogonal matrix whose columns are the eigenvectors of $JJ^T$. $V$ is another orthogonal matrix whose columns are the  eigenvectors of $J^{T}J$. $S$ is a diagonal matrix and is the same size as $J$. Its diagonal elements are the square roots of the nonzero eigenvalues of both $JJ^T$ and $J^{T}J$. The elements of $S$ are the singular values of $J$ and they are represented as $\sigma_1, \sigma_2, ..., \sigma_r$ where $r$ is the rank of $J$. After factoring $J$ into these three components the pseudo inverse of $J$, $J^+$ can be calculated using (\ref{eq.PINV}) \cite{PINV}
\begin{equation}\label{eq.PINV}
    J^+ = US^{+}V^T
\end{equation}
$S^+$ is calculated by taking the reciprocal of all the non-zero elements of $S$ and leaving all the zero elements alone \cite{PINV}.  

\subsection{Control Algorithm}
The core functionality of the algorithm proposed is based on the voltage sensitivity analysis and the concept of electrical distances calculation mentioned before. The main objective of the coordinated voltage control algorithm is the monitoring and control of the voltage at each node of the system, by using inverter-based reactive power and voltage control with traditional voltage regulation devices in order to keep the voltages between the range desired. The calculation of the reactive power needed to be injected is obtained by using the voltage sensitivity and electrical distances concepts. The main steps of the control algorithm is given below.

\begin{enumerate}
%1
\item \textbf{Get nodes effected:} When a voltage violation is detected. The algorithm starts by receiving the location of the affected nodes.
%2
\item \textbf{Determine node with highest voltage deviation.}
%3
\item \textbf{Construct system jacobian matrix}
%4
\item \textbf{Calculate Sensitivity matrix}
%5
\item \textbf{Calculate electrical distances}
%6
\item \textbf{Order the nodes:} In this step the nodes capable of providing reactive power support are ordered according to their electrical distance from the selected effected node. The ordered list of nodes is saved in a list called \textit{Node\_list}
%7
\item \textbf{Apply the appropriate reactive power:} In this step the first node in the \textit{Node\_list} is selected. If the node has the capability to supply the required reactive power determined by the sensitivity matrix the algorithm sets the node to provide the required amount of power and continues to \textbf{step 9}. Otherwise the algorithm sets the node to deliver the maximum available reactive power and takes it out of the list of nodes capable of providing reactive power.
%8
\item \textbf{Repeat steps 3 to 7}
%9
\item \textbf{Request the devices at the nodes to apply calculated changes}
\end{enumerate}

% \begin{figure}[h]
% \centering
% \includegraphics[width=0.5\textwidth]{cvc_alg.png}
% \caption{Coordinated Voltage Control Algorithm Flowchart}
% \label{fig:f1}
% \end{figure}
% The algorithm starts by receiving the needed information from the remote locations. It obtains the voltage and current measurements for each bus and the real and reactive power being generated or absorbed by the DGs. Based on these values, all the voltages are checked against the maximum and minimum voltage range criteria to detect under-voltage or over-voltage issues on specific nodes. When a voltage deviation is detected, the algorithm will start solving the issue from the furthest node to the closest one. The first step in solving this deviation is the application of voltage sensitivity concept to the affected node in order to find the approximate amount of reactive needed be injected at all nodes on control zone of the system. The voltage sensitivity process will return the Jacobian matrix for the system, Jvpq, and the reactive power that each node, in the control zone of the affected bus, needs to individually inject to solve the voltage deviation problem.  After this calculation is completed, a priority list of the nodes on control zone is created based on the criteria of the electrical distances calculated using the method mentioned above. The node closest to the affected node is selected and checked against the nodes that can inject reactive power into the system, in other words, the list of the nodes with DG connected to them. If the node selected to help in the voltage control has the capability of injecting reactive power, the value is checked against all other nodes to assure the compatibility of the value and make sure that no other voltage nodes get out of the desired bounds. As seen on the flowchart, if this value complies with the requirements mentioned, the reactive power reference value is sent to the respective DG inverter. If the value does not comply with the requirements, the next node in the electrical list is selected and the entire process is performed again. Another option for the solution of the reactive power reference calculation, is the application of the maximum reactive power that the current node can supply and the rerun of the entire algorithm to find the solution for the remaining voltage deviation. The algorithm is designed to handle both of these cases and try to minimize the impact in voltage caused by reactive power being injected or absorbed into the system. 




%\subsection{Test System}



\section{Test system}
\input{03_test_system.tex}

\section{Validation and Results}\label{sec:val}
\input{04_val_res.tex}

%\begin{enumerate}
%1
\item \textbf{Get nodes effected:} When a voltage violation is detected. The algorithm starts by receiving the location of the affected nodes.
%2
\item \textbf{Determine node with highest voltage deviation.}
%3
\item \textbf{Construct system jacobian matrix}
%4
\item \textbf{Calculate Sensitivity matrix}
%5
\item \textbf{Calculate electrical distances}
%6
\item \textbf{Order the nodes:} In this step the nodes capable of providing reactive power support are ordered according to their electrical distance from the selected effected node. The ordered list of nodes is saved in a list called \textit{Node\_list}
%7
\item \textbf{Apply the appropriate reactive power:} In this step the first node in the \textit{Node\_list} is selected. If the node has the capability to supply the required reactive power determined by the sensitivity matrix the algorithm sets the node to provide the required amount of power and continues to \textbf{step 9}. Otherwise the algorithm sets the node to deliver the maximum available reactive power and takes it out of the list of nodes capable of providing reactive power.
%8
\item \textbf{Repeat steps 3 to 7}
%9
\item \textbf{Request the devices at the nodes to apply calculated changes}
\end{enumerate}

\section{Conclusion}
As the penetration of distributed generation resources starts to increase in distribution networks, the deployment of a coordinated voltage scheme is a requirement most utilities will have to fulfill. As observed, the algorithm proposed in this paper has the ability to coordinate the reactive power reference points for the distributed generation and traditional voltage regulation devices located along the feeder, taking into account their potential impacts on the entire system. By using this approach, a reduction in the use of traditional regulation devices can be achieved and the voltage profiles along the network can be controlled for the most part by just using the already existing DG inverters present on the network. In future work this concept of control will be expanded to unbalanced networks.

\bibliographystyle{IEEEtran}
\bibliography{mybib.bib}

%%%%%%%%%%%%%%%%%%%%%%%%%%%%%%%%%%%%%%%%%%%%%%%%%

% \begin{thebibliography}{1}

% \bibitem{int1}
% D.~Ranamuka, A.~P. Agalgaonkar and K.~M. Muttaqi, "Online Coordinated Voltage Control in Distribution Systems Subjected to Structural Changes and DG Availability,"\emph{ IEEE Transactions on Smart Grid}, vol. 7, no. 2, pp. 580-591, 2016. 
% \bibitem{int2}
% W.~Sheng, K.~-y. Liu and S. Cheng, "A Trust Region SQP Method for Coordinated Voltage Control in Smart Distribution Grid," \emph{IEEE Transactions on Smart Grid}, vol. 7, no. 1, pp. 381 - 391, 2016. 
% \bibitem{Th_ali}
% A.~Hariri, "Simulation Tools and Techniques for Analyzing the Impacts of Photovoltaic System Integration," Tallahassee Florida State University, 2017. 

% \bibitem{int4}
%  Paul Brucke Reactive Power Control in Utility-Scale PV, Solarprofessional.com, 2014. [Online]. Available: http://solarprofessional.com/articles/design-installation/reactive-power-control-in-utility-scale-pv/page/0/5. [Accessed: 24- Apr- 2017].

% \bibitem{SG}
% R. Meeker, M. Steurer and M. Faruque, "High penetration solar PV deployment sunshine state grid initiative (SUNGRIN)," Center for Advanced Power Systems, Tallahassee, 2015.

% \end{thebibliography}

%%%%%%%%%%%%%%%%%%%%%%%%%%%%%%%%%%%%%%%%%%%%%%%%%%%%%%%%%%

\end{document}