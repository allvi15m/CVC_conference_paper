\begin{enumerate}
%1
\item \textbf{Identify the nodes with voltage limit violation from measured data}
%2
\item \textbf{Determine node with highest voltage deviation.}
%3
\item \textbf{Construct system Jacobian matrix}
%4
\item \textbf{Calculate voltage sensitivity matrix}
%5
\item \textbf{Calculate electrical distances}
%6
\item \textbf{Order the nodes:} In this step, the nodes capable of providing reactive power support are ordered according to their electrical distance from the selected affected node. The ordered list of nodes is saved in a list called \textit{Node\_list}
%7
\item \textbf{Apply the appropriate reactive power:} In this step, the first node in the \textit{Node\_list} is selected. If the node has the capability to supply the required reactive power determined by the sensitivity matrix the algorithm sets the node to provide the required amount of power and continues to \textbf{step 9}. Otherwise, the algorithm sets the node to deliver the maximum available reactive power and takes it out of the list of nodes capable of providing reactive power.
%8
\item \textbf{Repeat steps 3 to 7}
%9
\item \textbf{Estimate new system status:} After deciding the change in node reactive powers the algorithm estimates the new voltage profile of the system using (\ref{eq.Q_V}). If all the nodes in the new estimated profile are within voltage limits the algorithm continues to \textbf{step 10}. Otherwise the algorithm updates the measured data collected from the system with the new estimates and goes back to  \textbf{step 1}.

%10
\item \textbf{Request the devices at the nodes to apply calculated changes}
\end{enumerate}